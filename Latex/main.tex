\begin{abstract}
Through this manual, we learn how to measure an unknown resistance through Vaman-ESP and display it on an LCD.
\end{abstract}
\subsection{Components}
\numberwithin{equation}{enumi}
\numberwithin{figure}{enumi}
%\numberwithin{table}{enumi}
\numberwithin{table}{section}


\begin{table}[!h]
\centering
\input{./vaman/vaman-esp/lcd/figs/components.tex}
\caption{Components}
\label{table:components}
\end{table}

\subsection{Setting up the Display}
\begin{enumerate}[label=\thesection.\arabic*.,ref=\thesection.\theenumi]
\numberwithin{equation}{enumi}
\numberwithin{figure}{enumi}
\numberwithin{table}{enumi}

	

\item
Plug the LCD in Fig. \ref{fig:lcd} to the breadboard.

%
\begin{figure}
\centering
\includegraphics[width=\columnwidth]{./vaman/vaman-esp/lcd/figs/lcd.eps}
\caption{LCD pins}
\label{fig:lcd}
\end{figure}


%
%\item
%Connect the 220$\Omega$ resistance from $V_{cc}$ to pin 15 (Led+) of the LCD.

%
%%
%\item
%Connect  pin 16 (Led-) of the LCD to GND.  The LCD should glow.
%
%%
%\item
%Connect pin 3 of the LCD to GND.  This is required for contrast.
%
\item
Connect the Vaman-ESP pins to LCD pins as per Table \ref{Table:1}. Make sure that all 5V sources are connected to the LCD through a 220 $\Omega$ resistance.
\item
The Vaman pin diagram is available in Fig. \ref{fig:vaman-pin_sheet}
\begin{table}[h]
\centering
\input{./vaman/vaman-esp/lcd/figs/table1.tex}
\caption{Make sure that all 5V sources are connected to the LCD through a 220 $\Omega$ resistance.}
\label{Table:1}
\end{table}

\item Execute the following code after editing the wifi credentials

\begin{lstlisting}
vaman/vaman-esp/lcd/codes/setup
\end{lstlisting}
You should see the following  message
\begin{lstlisting}
Hi
This is CSP Lab
\end{lstlisting}
\item Modify the above code to display your name.
\end{enumerate}
\subsection{Measuring the resistance}
\begin{enumerate}[label=\thesection.\arabic*.,ref=\thesection.\theenumi]
\numberwithin{equation}{enumi}
\numberwithin{figure}{enumi}
\numberwithin{table}{enumi}

\item
Connect the 5V pin of the Vaman-ESP to an extreme pin of the Breadboard shown in Fig. \ref{fig:breadboard}.  Let this pin be $V_{cc}$.

%
%
\begin{figure}[!ht]
\centering
\includegraphics[width=\columnwidth]{./vaman/vaman-esp/lcd/figs/breadboard.eps}
\caption{Breadboard}
\label{fig:breadboard}
\end{figure}
%
\item
Connect the GND pin of the Vaman-ESP to the opposite extreme pin of the Breadboard.

%
%
\item
Let $R_1$ be the known resistor and $R_2$ be the unknown resistor.  Connect $R_1$ and $R_2$ in series such that $R_1$ is connected
to $V_{cc}$ and $R_2$ is connected to GND. Refer to Fig. \ref{fig:voltage_divider}

%
%
\begin{figure}[!ht]
\centering
%\includegraphics[width=\columnwidth]{./figs/voltage_divider.eps}
\resizebox {\columnwidth} {!} {
%\input{./figs/collpits.tex}
\input{./vaman/vaman-esp/lcd/figs/vrr.tex}
}
%\input{./figs/vrr.tex}
%\includegraphics[width=\columnwidth]{./figs/voltage_divider.eps}
\caption{Voltage Divider}
\label{fig:voltage_divider}
\end{figure}
%
\item
Connect the junction between the two resistors to  the GPIO34 pin on the Vaman-ESP.

%
\item
Connect the Vaman-ESP to the computer so that it is powered.
%
%
%
%\item
%The resistance is an analogue function,but the value displayed on LCD is digital function.So,we need to do analogue to digital conversion,ESP32 has built-in 10-bit analogue to digital converter.
%Two resistors are used, one of which is known resistor value and other is unknown resistor value.
%Take 2 resistors,one known resistor with resistance value(For e.g-1K,2K,10K),here we have taken 1K and another an unknown resistor whose value we are going to calculate.
%
%
%

\item
Execute the following code after editing the wifi credentials

\begin{lstlisting}
vaman/vaman-esp/lcd/codes/resistance
\end{lstlisting}
%\solution 
%%
%\lstinputlisting{./codes/ESP32 _LCD/src/main.cpp}

\end{enumerate}
\subsection{Displaying the Measured resistance on LCD and website}
\begin{enumerate}[label=\thesection.\arabic*.,ref=\thesection.\theenumi]
\numberwithin{equation}{enumi}
\numberwithin{figure}{enumi}
\numberwithin{table}{enumi}
\item The unknown resistance is measured and diplayed the measured resistance on the LCD display and also on the Vaman-ESP webserver.
\item Connect the Vaman-ESP pins to LCD pins as per Table \ref{Table:1}.
\item
Execute the following code after editing the wifi credentials

\begin{lstlisting}
vaman/vaman-esp/lcd/webserver/codes
\end{lstlisting}
\item After flashing the code to vaman-ESP, the board will be connected to the wifi credentials provided.
\item Now connect the same WiFi credentials to the mobile phone for accessing the IP address, which can be accessed by 
\begin{lstlisting}
ifconfig
nmap -sn 192.168.x.x/24
\end{lstlisting}
\item Change the IP address in the second command accordingly with the IP address provided by first command.
\item By the above commands the IP address of vaman-ESP will be diplayed.
\item Now the vaman-ESP will be hosting a webserver
\item Inorder to access the webserver type the IP address of the vaman-ESP in the web browser.
\item In the website loaded by the IP address of vaman-ESP the Unknown resitance is displayed as shown in Fig. \ref{fig:website}

\begin{figure}[!ht]
\centering
\includegraphics[width=\columnwidth]{./vaman/vaman-esp/lcd/figs/website.jpg}
\caption{Website}
\label{fig:website}
\end{figure}
\end{enumerate}

\subsection{Explanation}
\begin{enumerate}[label=\thesection.\arabic*.,ref=\thesection.\theenumi]
\numberwithin{equation}{enumi}
\numberwithin{figure}{enumi}
\numberwithin{table}{enumi}

%\begin{enumerate}

\item We create a variable called analogPin and assign it to 0. 
This is because the voltage value we are going to read is connected to analogPin GPIO34.

\item  The 12-bit ADC can differentiate 4096 discrete voltage levels, 5 volt is applied to 2 resistors and the voltage sample is taken in between the resistors. The value which we get from analogPin can be between 0 and 4095. 0 would represent 0 volts falls across the unknown resistor. A value of 4095 would mean that practically all 5 volts falls across the unknown resistor.

\item  $V_{out}$ represents the divided voltage that falls across the unknown resistor.

\item  The Ohm meter in this manual works on the principle of the voltage divider shown in Fig. \ref{fig:voltage_divider}.
%
\begin{align}
V_{out}&=\frac{R_1}{R_1+R_2}V_{in} \\
\Rightarrow R_2&=R_1\brak{\frac{V_{in}}{V_{out}}-1}
\end{align}
%
In the above, $V_{in} = 5$V, $R_1 = 220 \Omega$.
\item Repeat the exercise with another unknown resistance.
\end{enumerate}
